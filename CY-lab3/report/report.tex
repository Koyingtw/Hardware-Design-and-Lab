%% 该模板修改自《计算机学报》latex 模板
%% 主要是将双栏改成单栏,去掉了部分计算机学报标识;
%% 源文件自:https://www.overleaf.com/latex/templates/latextemplet-cjc-xelatex/ybmmymncrrmw
%% 
%%
%% This is file `CjC_template_tex.tex',
%% is modified by Zhi Wang (zhiwang@ieee.org) based on the template 
%% provided by Chinese Journal of Computers (http://cjc.ict.ac.cn/).
%%
%% This version is capable with Overleaf (XeLaTeX).
%%
%% Update date: 2023/03/10
%% -------------------------------------------------------------------
%% Copyright (C) 2016--2023 
%% -------------------------------------------------------------------
%% This file may be distributed and/or modified under the
%% conditions of the LaTeX Project Public License, either version 1.3c
%% of this license or (at your option) any later version.
%% The latest version of this license is in
%%    https://www.latex-project.org/lppl.txt
%% and version 1.3c or later is part of all distributions of LaTeX
%% version 2008 or later.
%% -------------------------------------------------------------------

\documentclass[10.5pt,compsoc,UTF8]{CjC}
\usepackage{CTEX}
\usepackage{graphicx}
\usepackage{footmisc}
\usepackage{subfigure}
\usepackage{url}
\usepackage{multirow}
\usepackage{multicol}
\usepackage[noadjust]{cite}
\usepackage{amsmath,amsthm}
\usepackage{amssymb,amsfonts}
\usepackage{booktabs}
\usepackage{color}
\usepackage{ccaption}
\usepackage{booktabs}
\usepackage{float}
\usepackage{fancyhdr}
\usepackage{caption}
\usepackage{xcolor,stfloats}
\usepackage{comment}
\setcounter{page}{1}
\graphicspath{{figures/}}
\usepackage{cuted}%flushend,
\usepackage{captionhack}
\usepackage{epstopdf}
\usepackage{gbt7714}
\usepackage{listings}
\usepackage{xeCJK}
\usepackage{float}
\usepackage{sourcecodepro}
\usepackage[T1]{fontenc}
\usepackage{hyperref}

\setmainfont{Times Roman}
% \setCJKmainfont{Noto Sans Mono CJK TC}
\setCJKmainfont{標楷體.ttc}
\setmonofont{Cascadia Code}

%===============================%

\headevenname{\mbox{\quad} \hfill  \mbox{\zihao{-5}{ \hfill 2024 Hardware Design  } \hspace {50mm} \mbox{2024 年 2 月}}}%
\headoddname{Group 21 \hfill Lab 1: Gate-Level Verilog}%

%footnote use of *
\renewcommand{\thefootnote}{\fnsymbol{footnote}}
\setcounter{footnote}{0}
\renewcommand\footnotelayout{\zihao{5-}}

\newtheoremstyle{mystyle}{0pt}{0pt}{\normalfont}{1em}{\bf}{}{1em}{}
\theoremstyle{mystyle}
\renewcommand\figurename{figure~}
\renewcommand{\thesubfigure}{(\alph{subfigure})}
\newcommand{\upcite}[1]{\textsuperscript{\cite{#1}}}
\renewcommand{\labelenumi}{(\arabic{enumi})}
\newcommand{\tabincell}[2]{\begin{tabular}{@{}#1@{}}#2\end{tabular}}
\newcommand{\abc}{\color{white}\vrule width 2pt}
\renewcommand{\bibsection}{}
\makeatletter
\renewcommand{\@biblabel}[1]{[#1]\hfill}
\makeatother
\setlength\parindent{2em}
%\renewcommand{\hth}{\begin{CJK*}{UTF8}{gbsn}}
%\renewcommand{\htss}{\begin{CJK*}{UTF8}{gbsn}}
\renewcommand{\contentsname}{Table of Contents}

\begin{document}

\hyphenpenalty=50000
\makeatletter
\newcommand\mysmall{\@setfontsize\mysmall{7}{9.5}}
\newenvironment{tablehere}
  {\def\@captype{table}}

\let\temp\footnote
\renewcommand \footnote[1]{\temp{\zihao{-5}#1}}

\hypersetup{
  colorlinks=false,
  pdfborder={0 0 0},
}

\thispagestyle{plain}%
\thispagestyle{empty}%
\pagestyle{CjCheadings}

% \begin{table*}[!t]
\vspace {-13mm}


\onecolumn
\zihao{5-}\noindent Group 21 \hfill Lab 3: Sequential Circuits \hfill 2024 年 10 月\\
\noindent\rule[0.25\baselineskip]{\textwidth}{1pt}


\begin{center}
    \vspace {11mm}
    {\zihao{2} \heiti \fangsong Lab 3: Sequential Circuits }
    
    \vskip 5mm
    
    {\zihao{4}\fangsong Group 21: 陳克盈(112062205)、蔡明妡(112062224)}
\end{center}

\lstset{
    % backgroundcolor=\color{red!50!green!50!blue!50},%程式碼塊背景色為淺灰色
    rulesepcolor= \color{gray}, %程式碼塊邊框顏色
    breaklines=true,  %程式碼過長則換行
    numbers=left, %行號在左側顯示
    numberstyle= \small\ttfamily,%行號字型
    keywordstyle= \color{blue},%關鍵字顏色
    commentstyle=\color{gray}, %註釋顏色
    frame=shadowbox%用方框框住程式碼塊
    basicstyle=\ttfamily\footnotesize,
}
 
\definecolor{improvecolor}{rgb}{0,0.6,0} % 深綠色
\definecolor{declinecolor}{rgb}{0.6,0,0} % 深紅色


%%%%%%%%%%%%%%%%%%%%%%%%%%%%%%%%%%%%%%
\zihao{5}
\vskip 10mm
% \begin{multicols}{1}


%%%%%%%%%%%%%%%%%%%%%%%%%%%%%%%%%%%%%%%%%%
%%%%%%%%%%%%%%%%%%%%%%%%%%%%%%%%%%%%%%%%%%

\tableofcontents
\newpage

\section{Q1: 4-bit Ping-Pong Counter}
\label{sec:Q1}

\subsection{Implement}
這題需要實作一個 4-bit 的 Ping-Pong Counter,會從 0 開始累加\
,直到 15 後再從 15 開始減少至 0,不斷循環。

在 verilog 程式碼方面,我們使用了多個 if 條件式,判斷目前屬於什麼狀態:
\begin{itemize}
  \item $rst\_n = 0$: 將 $started, direction$ 設為 true, 並將輸出 $out$ 設為 $0$。
  \item $started \& enable$:代表 Counter 正在運作,此時會根據 $direction$ 進行加減:
  \begin{itemize}
    \item $out = 15, direction = 1$:代表再繼續往上就會超出範圍,因此將 $direction$ 設為 $0$ 並將輸出 $-1$
    \item $out = 0, direction = 0$:與上個情況相反,將 $direction$ 設為 $1$ 並將輸出 $+1$
    \item 其他情況:根據 $direction$ 進行加減
  \end{itemize}
\end{itemize}

這裡要放 code 截圖

\subsection{Circuit}

在電路方面,需要實作的部分如下:
\begin{itemize}
  \item $started, direction, out$ 儲存:使用了三個 D-Flip-Flop 來儲存
  \item 計算下一個 clock cycle 的 $out, direction$:每個 clock cycle 會根據當下的 $out, direction$,透過加減法器預先算出 $out + 1, out - 1$,\
  再透過多個 AND gate, NOT Gate,來實作出上述的條件式,分別導至四條線路。\
  最後,再利用 MUX,根據對應的線路與條件來決定下一個 clock cycle 的 $out$ 以及 $direction$。
  \item enable:算出 $out, direction$ 之後,利用 MUX 處理 $enable$,若是 $enable = True$,則將 新的 $out, direction$ 輸入至 D-Flip-Flop,\
  否則就輸入原來的值。
  \item reset:在上述訊號都處理完後,會再經過一關判斷是否需要 reset,如果不用則使用上述計算出來的值,否則就將該值設為初始值。
\end{itemize}

這裡要放電路圖

\subsection{Testbench}
由於操作相較單純,因此直接利用一個迴圈跑 $2^8$ 次,每次都將 enable 反轉,觀察輸出是否符合預期:

這裡放程式碼截圖

這裡放波形圖部分

這裡放波形圖全貌

\section{Q2: First-In First Out (FIFO) Queue}
\label{sec:Q2}

\subsection{Implement}

這邊要實現一個 8 個 8-bit 資料的 Queue (First in First out),\
實作內容主要以這幾個數值的計算為主:
\begin{itemize}
  \item $started$:是否開始運作,為 register,會在每個 posedge 時被計算
  \item $raddr$:讀取的位址,為 register,會在每個 posedge 時被計算
  \item $wadd$:下一次要寫入的位址,為 register,會在每個 posedge 時被計算
  \item $count$:目前 Queue 中有幾個資料,為 register,會在每個 posedge 時被計算
  \item $error$:是否有錯誤,為 wire,判斷方式為:\
  當 $count = 0$ 且 $ren = 0$,或是 $count = 8$ 且 $wen = 1$ 時,error 為 True,\
  前者代表沒有資料可以讀取,後者代表沒有空間可以寫入。
\end{itemize}

而操作過程如下:
\begin{enumerate}
  \item reset:當 $rst\_n$ 與 $started$ 皆為 False 時,將 $started$ 設為 True
  \item 開始後,如果沒有錯誤且 $ren = 1$,則將輸出設定為 $raddr$ 的位置、將 $raddr$ 設為 $raddr + 1$,\
  以及將 $count$ 減一
  \item 如果上述條件都沒有達到,且要寫入的話,則將 $waddr$ 的位置設定為 $din$ (data input)、\
  將 $waddr$ 設為 $waddr + 1$,以及將 $count$ 加一
\end{enumerate}

以此邏輯實現的程式碼就會如下圖所示:

這裡放程式碼截圖

\subsection{Circuit}

\subsubsection*{Memory}
首先先介紹我們設計的記憶體單位,我們透過 8 個 8-bit 的 D-Flip-Flop 作為儲存單位,\ 
而記憶體控制的部分,除了 clock 之外會接收 4 個參數,分別為:
\begin{itemize}
  \item $ren$:是否讀取記憶體
  \item $wen$:是否寫入記憶體
  \item $addr$:讀取或寫入的記憶體位置
  \item $din$:寫入的資料
\end{itemize}
並且會輸出 $dout$,代表讀取的資料。
\par
判斷邏輯會是這樣的步驟:
\begin{enumerate}
  \item 如果 $ren = 1$,那麼不執行寫入,直接輸出 $addr$ 這個位置的值
  \item 如果沒有要讀取,且要寫入的話,那麼就將 $din$ 的值設為是 $addr$ 這個位置新的值
  \item 沒有要讀取也沒有要寫入的話,就輸出 $0$ 然後不對 DFF 做任何動作
\end{enumerate}

這裡放程式碼截圖

在電路實現的部分,大致會分成幾層:
\begin{enumerate}
  \item 判斷輸入:首先使用比較器對每個位址 $i$ 判斷 $addr$ 是否等於 $i$,\
  並將其結果與 $!ren \& wen$ 做 AND,得到的結果便是這個位址是否要寫入。
  \item D-Flip-Flop 控制:在每個 DFF 前加上一個 MUX,根據上述的結果決定 DFF 的輸入是\
  $din$ 還是原本的值
  \item 輸出判斷:與第一步相似但相較簡單,直接使用比較器判斷 $addr$ 是否等於 $i$,\
  並將結果與 DFF 的輸出做 AND,就能得到這個位址的輸出值。由於只會有一個位址會有輸出值,\
  因此在最後直接做 Bitwise OR 就可以得到最終的輸出。
\end{enumerate}

下圖演示的是 8 個 8-bit 的記憶體,可根據需求調整每個單位的位元大小以及有幾個單位。

這裡放電路圖

\subsubsection*{FIFO}
FIFO 的部分則是利用了上述的記憶體單位,計算出要傳給 Memory 的 $ren, wen, addr, din$ 後,得到輸出的結果。

這裡放電路圖解釋

這裡放電路圖截圖

\subsection{Testbench}
我們寫了兩個 Testbench,首先是重現題目範例中的波形圖:

% TODO: 這裡放題目範例截圖

接著是我們使用 SystemVerilog 提供的 Queue,進行更完整的測試。\
由於 $ren, wen, din$ 的組合數太多,因此我選擇了使用跑 $2^10$ 測試,每次將這些輸入值設定為一個隨機值,\
藉此來跑到盡可能多的 case,並透過與 SystemVerilog 中,絕對正確的 Queue 來比對是否有錯誤。

% TODO: 這裡放程式碼截圖

\section{Q3: Multi-Bank Memory}
\label{sec:Q3}

\subsection{Implement}

這題需要實作一個 Multi-Bank Memory,每個 Bank 有 4 個 Sub-Bank,\
每個 Sub-Bank 由 128 個 8-bit memory 組成,總共有 4 個 Bank。

輸入的部分由以下幾個部分組成:
\begin{itemize}
  \item $ren, wen$:是否讀取或寫入
  \item $raddr$(11 bit):讀取的位址,前兩個 Bit 代表著是第幾個 Bank,接下來的兩個代表的是第幾個 Subbank。
  \item $waddr$ (11 bit):寫入的位址,格式同上
  \item $din$ (8 bit):寫入的資料
\end{itemize}

% TODO: 這裡放題目截圖


\subsubsection*{Bank}


\subsubsection*{主 Module}
主要的 Module 負責整體的輸入輸出,將輸入的參數正確的分配給各個 Bank,\
並將正確的 Bank 輸出導向至最後的輸出。

在處理輸入的部分,主要分為以下步驟:
\begin{itemize}
  \item $ren$:直接利用 $ren \& (raddr[10:9]) == i\ (i \in [0, 3])$ 的方式來決定每個 Bank 的 $ren$
  \item $wen$:分為兩個步驟
  \begin{enumerate}
    \item 判斷是否能夠輸入,題目規定,當要求同時讀寫同一個 Sub-bank 時,就以讀的操作為主,因此先利用\
    \texttt{wen \&\& (ren == 0 || (waddr[10:7] != raddr[10:7]))} 判斷出這次的寫入操作是否合法
    \item 與 $ren$ 用一樣的方式將上述的信號分配到正確的 Bank
  \end{enumerate}
\end{itemize}

至於輸出的部分,我使用了四個 8 bit 的 wire: $out[3:0]$,分別接至四個 Bank 的輸出,\
由於在處理輸入信號的部分,當該 Bank 不是要求的 Bank 時,output 會直接是 $0$,\
因此要取得正確的輸出就只要將四個 output 做 Bitwise OR 即可。

% TODO: 這邊放程式碼截圖

這部分

由於 Memory 的部分在 Basic 已經有實作過,因此這邊就直接沿用 Basic 的設計。

\section{Q4: Round-Robin FIFO Arbiter}

\section{Q5: 4-bit Parameterized Ping-Pong Counter}

\end{document}


