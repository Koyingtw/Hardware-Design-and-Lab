%% 该模板修改自《计算机学报》latex 模板
%% 主要是将双栏改成单栏,去掉了部分计算机学报标识;
%% 源文件自:https://www.overleaf.com/latex/templates/latextemplet-cjc-xelatex/ybmmymncrrmw
%% 
%%
%% This is file `CjC_template_tex.tex',
%% is modified by Zhi Wang (zhiwang@ieee.org) based on the template 
%% provided by Chinese Journal of Computers (http://cjc.ict.ac.cn/).
%%
%% This version is capable with Overleaf (XeLaTeX).
%%
%% Update date: 2023/03/10
%% -------------------------------------------------------------------
%% Copyright (C) 2016--2023 
%% -------------------------------------------------------------------
%% This file may be distributed and/or modified under the
%% conditions of the LaTeX Project Public License, either version 1.3c
%% of this license or (at your option) any later version.
%% The latest version of this license is in
%%    https://www.latex-project.org/lppl.txt
%% and version 1.3c or later is part of all distributions of LaTeX
%% version 2008 or later.
%% -------------------------------------------------------------------

\documentclass[10.5pt,compsoc,UTF8]{CjC}
\usepackage{CTEX}
\usepackage{graphicx}
\usepackage{footmisc}
\usepackage{subfigure}
\usepackage{url}
\usepackage{multirow}
\usepackage{multicol}
\usepackage[noadjust]{cite}
\usepackage{amsmath,amsthm}
\usepackage{amssymb,amsfonts}
\usepackage{booktabs}
\usepackage{color}
\usepackage{ccaption}
\usepackage{booktabs}
\usepackage{float}
\usepackage{fancyhdr}
\usepackage{caption}
\usepackage{xcolor,stfloats}
\usepackage{comment}
\setcounter{page}{1}
\graphicspath{{figures/}}
\usepackage{cuted}%flushend,
\usepackage{captionhack}
\usepackage{epstopdf}
\usepackage{gbt7714}
\usepackage{listings}
\usepackage{xeCJK}
\usepackage{float}
\usepackage{sourcecodepro}
\usepackage[T1]{fontenc}
\usepackage{hyperref}

\setmainfont{Times Roman}
% \setCJKmainfont{Noto Sans Mono CJK TC}
\setCJKmainfont{標楷體.ttc}
\setmonofont{Cascadia Code}

%===============================%

\headevenname{\mbox{\quad} \hfill  \mbox{\zihao{-5}{ \hfill 2024 Hardware Design  } \hspace {50mm} \mbox{2024 年 2 月}}}%
\headoddname{Group 21 \hfill Lab 1: Gate-Level Verilog}%

%footnote use of *
\renewcommand{\thefootnote}{\fnsymbol{footnote}}
\setcounter{footnote}{0}
\renewcommand\footnotelayout{\zihao{5-}}

\newtheoremstyle{mystyle}{0pt}{0pt}{\normalfont}{1em}{\bf}{}{1em}{}
\theoremstyle{mystyle}
\renewcommand\figurename{figure~}
\renewcommand{\thesubfigure}{(\alph{subfigure})}
\newcommand{\upcite}[1]{\textsuperscript{\cite{#1}}}
\renewcommand{\labelenumi}{(\arabic{enumi})}
\newcommand{\tabincell}[2]{\begin{tabular}{@{}#1@{}}#2\end{tabular}}
\newcommand{\abc}{\color{white}\vrule width 2pt}
\renewcommand{\bibsection}{}
\makeatletter
\renewcommand{\@biblabel}[1]{[#1]\hfill}
\makeatother
\setlength\parindent{2em}
%\renewcommand{\hth}{\begin{CJK*}{UTF8}{gbsn}}
%\renewcommand{\htss}{\begin{CJK*}{UTF8}{gbsn}}
\renewcommand{\contentsname}{Table of Contents}

\begin{document}

\hyphenpenalty=50000
\makeatletter
\newcommand\mysmall{\@setfontsize\mysmall{7}{9.5}}
\newenvironment{tablehere}
  {\def\@captype{table}}

\let\temp\footnote
\renewcommand \footnote[1]{\temp{\zihao{-5}#1}}

\hypersetup{
  colorlinks=false,
  pdfborder={0 0 0},
}

\thispagestyle{plain}%
\thispagestyle{empty}%
\pagestyle{CjCheadings}

% \begin{table*}[!t]
\vspace {-13mm}


\onecolumn
\zihao{5-}\noindent Group 21 \hfill Lab 1: Gate-Level Verilog \hfill 2024 年 2 月\\
\noindent\rule[0.25\baselineskip]{\textwidth}{1pt}


\begin{center}
    \vspace {11mm}
    {\zihao{2} \heiti \fangsong Lab 1: Gate-Level Verilog }
    
    \vskip 5mm
    
    {\zihao{4}\fangsong Group 21: 陳克盈(112062205)、蔡明妡(112062224)}
\end{center}

\lstset{
    % backgroundcolor=\color{red!50!green!50!blue!50},%程式碼塊背景色為淺灰色
    rulesepcolor= \color{gray}, %程式碼塊邊框顏色
    breaklines=true,  %程式碼過長則換行
    numbers=left, %行號在左側顯示
    numberstyle= \small\ttfamily,%行號字型
    keywordstyle= \color{blue},%關鍵字顏色
    commentstyle=\color{gray}, %註釋顏色
    frame=shadowbox%用方框框住程式碼塊
    basicstyle=\ttfamily\footnotesize,
}
 
\definecolor{improvecolor}{rgb}{0,0.6,0} % 深綠色
\definecolor{declinecolor}{rgb}{0.6,0,0} % 深紅色


%%%%%%%%%%%%%%%%%%%%%%%%%%%%%%%%%%%%%%
\zihao{5}
\vskip 10mm
% \begin{multicols}{1}


%%%%%%%%%%%%%%%%%%%%%%%%%%%%%%%%%%%%%%%%%%
%%%%%%%%%%%%%%%%%%%%%%%%%%%%%%%%%%%%%%%%%%

\tableofcontents
\newpage

\section{NAND Gate to all other gates}
\label{sec:NAND Gate to all other gates}
此章節利用 NAND 所組成的所有邏輯閘,都將用來作為後續所有題目的邏輯閘使用。\\
在表示上為求可讀性,將會使用一般的邏輯閘符號來表示。



\section{Full Adder vs. Half Adder}

Half Adder 雖然能夠算出總和以及進位值,但由於缺少了 $cin$  的輸入,\\
導致他只能處理單一位元的加法,這也是為什麼他被稱作是 Half Adder。

\section{Q1: 8-bit ripple carry adder}
\label{sec:Q1}

根據題目所求,建立八個 Basic Q3 撰寫的 Full Adder,並將他們依據順序將輸入輸出串接在一起,\\
便完成了 8-bit ripple carry adder。

\section{Q2: Decode and execute}

\subsection{Universal gate}
Universal gate 由 $a \& !b$ 組成,不難發現,當我們將 $a$ 接上 True 信號,\
那麼這個 Gate 就會變成一個 NOT Gate,再利用這個 NOT Gate 接到 $b$ 上,\
就變成了一個 AND Gate。最後,將這個 NOT Gate 接到 AND Gate 的輸出上,\
就完成了一個 NAND Gate。

剩下的所有基本邏輯閘都可以透過這個生成出來的 NAND Gate 組合出來。\
由於組合方法已經在章節\ref{sec:NAND Gate to all other gates}中提到,\
因此這裡就不再贅述。

\subsection{Executor}
\subsubsection{SUB}
首先,$rs - rt$ 可以被轉換為 $rs + (-rt)$,根據二補數的規則,可以再轉換為 $rs + \sim rt + 1$。\
再搭配章節\ref{sec:Q1}實現的加法器,改變成 $4-bit Adder$ 後,將 $(a, b, cin)$ 設為 $rs, \sim rt, 1$,便是 SUB 的實現。

\subsubsection{ADD}
ADD 的實現同樣基於章節\ref{sec:Q1}實現的加法器,將 $(a, b, cin)$ 設為 $rs, rt, 0$,便是 ADD 的實現。

\subsubsection{BITWISE OR}
利用 OR Gate 將每個位元分別運算。
這裡放圖就好

\subsubsection{BITWISE AND}
利用 AND Gate 將每個位元分別運算。

這裡也是放圖就好

\subsubsection{RIGHT SHIFT}
由於一個 4-bit 的數字右移一位後,$out_0 \sim out_3$ 分別會對應到 $in_1, in_2, in_3, in_0$ \
因此只需要透過與 $1$ 做 AND 能夠實現 assign 的特性,將輸出對應到正確的輸入即可。

\subsubsection{LEFT SHIFT}
左移實作方式與右移相同,唯一的區別在於 $out_0 \sim out_3$ 分別對應到的是 $in_3, in_0, in_1, in_2$。

\subsection{Decoder}
首先將 Executor 生成的結果依 $OP\_Code$ 順序導至 $0 \sim 7$,由於指令只有八種,\
因此我們可以使用七個 2-to-1 mux,並分成三個階段來實現 Decoder。

\subsubsection{COMPARE LT}
由於比較的結果決定於由高位數來第一個不同的位元,因此一個 4-bit 的比較小於將基於以下步驟:
\begin{enumerate}
  \item 比較 $rs[3]$ 是否 $< rt[3]$
  \item 若上個步驟為 False 且 $rs[3] = rt[3]$,則比較 $rs[2]$ 是否 $< rt[2]$
  \item 重複上述步驟,直到找到第一個不同的位元
\end{enumerate}

這幾個步驟可以寫成一個表達式:
$out = (rs[3] < rt[3]) \mid (rs[3] = rt[3] \& rs[2] < rt[2]) \mid (rs[3] = rt[3] \& \
rs[2] = rt[2] \& rs[1] < rt[1]) \mid (rs[3] = rt[3] \& rs[2] = rt[2] \& rs[1] = rt[1] \& rs[0] < rt[0])$

根據以上式子實作便可得到比較的結果,最後再加上題目規定的 $1010_{(2)}$ 即可。

\begin{enumerate}
  \item 第一階段,將 $sel[0]$ 作為 MUX 的 select bit,並依照 $OP\_Code$ \
  分成 $(0, 1), (2, 3), (4, 5), (6, 7)$,各自輸入到一個 MUX 中,便可將八種指令篩選成四種 
  \item 第二階段,與第一階段相同,將 $sel[1]$ 作為 MUX 的 select bit,便可以將四種指令再篩選成兩種。
  \item 第三階段,將 $sel[2]$ 作為 MUX 的 select bit,便可以得到我們所需要的輸出。
\end{enumerate}

\subsubsection{COMPARE EQ}
兩個 bit 要相等只有兩種可能:$0, 0$ 或是 $1, 1$,因此 $a = b$ 可以利用 $(!a \& !b) \mid (a \& b)$ 來表示。\\

分別對四個位元做以上操作,並確認是否都為 True,最終加上題目要求的 $1110_{(2)}$ 即為答案。

\section{Q3: 4-bit carry-lookahead adder}
CLA 的運算過程可以分為:PG Generator, CLA Generator, Adder 三個部分:

\subsection{PG Generator}
首先定義

\end{document}


