%% 该模板修改自《计算机学报》latex 模板
%% 主要是将双栏改成单栏,去掉了部分计算机学报标识;
%% 源文件自:https://www.overleaf.com/latex/templates/latextemplet-cjc-xelatex/ybmmymncrrmw
%% 
%%
%% This is file `CjC_template_tex.tex',
%% is modified by Zhi Wang (zhiwang@ieee.org) based on the template 
%% provided by Chinese Journal of Computers (http://cjc.ict.ac.cn/).
%%
%% This version is capable with Overleaf (XeLaTeX).
%%
%% Update date: 2023/03/10
%% -------------------------------------------------------------------
%% Copyright (C) 2016--2023 
%% -------------------------------------------------------------------
%% This file may be distributed and/or modified under the
%% conditions of the LaTeX Project Public License, either version 1.3c
%% of this license or (at your option) any later version.
%% The latest version of this license is in
%%    https://www.latex-project.org/lppl.txt
%% and version 1.3c or later is part of all distributions of LaTeX
%% version 2008 or later.
%% -------------------------------------------------------------------

\documentclass[10.5pt,compsoc,UTF8]{CjC}
\usepackage{CTEX}
\usepackage{graphicx}
\usepackage{footmisc}
\usepackage{subfigure}
\usepackage{url}
\usepackage{multirow}
\usepackage{multicol}
\usepackage[noadjust]{cite}
\usepackage{amsmath,amsthm}
\usepackage{amssymb,amsfonts}
\usepackage{booktabs}
\usepackage{color}
\usepackage{ccaption}
\usepackage{booktabs}
\usepackage{float}
\usepackage{fancyhdr}
\usepackage{caption}
\usepackage{xcolor,stfloats}
\usepackage{comment}
\setcounter{page}{1}
\graphicspath{{figures/}}
\usepackage{cuted}%flushend,
\usepackage{captionhack}
\usepackage{epstopdf}
\usepackage{gbt7714}
\usepackage{listings}
\usepackage{xeCJK}
\usepackage{float}
\usepackage{sourcecodepro}
\usepackage[T1]{fontenc}
\usepackage{hyperref}

\setmainfont{Times Roman}
% \setCJKmainfont{Noto Sans Mono CJK TC}
\setCJKmainfont{標楷體.ttc}
\setmonofont{Cascadia Code}

%===============================%

\headevenname{\mbox{\quad} \hfill  \mbox{\zihao{-5}{ \hfill 2024 Hardware Design  } \hspace {50mm} \mbox{2024 年 2 月}}}%
\headoddname{Group 21 \hfill Lab 1: Gate-Level Verilog}%

%footnote use of *
\renewcommand{\thefootnote}{\fnsymbol{footnote}}
\setcounter{footnote}{0}
\renewcommand\footnotelayout{\zihao{5-}}

\newtheoremstyle{mystyle}{0pt}{0pt}{\normalfont}{1em}{\bf}{}{1em}{}
\theoremstyle{mystyle}
\renewcommand\figurename{figure~}
\renewcommand{\thesubfigure}{(\alph{subfigure})}
\newcommand{\upcite}[1]{\textsuperscript{\cite{#1}}}
\renewcommand{\labelenumi}{(\arabic{enumi})}
\newcommand{\tabincell}[2]{\begin{tabular}{@{}#1@{}}#2\end{tabular}}
\newcommand{\abc}{\color{white}\vrule width 2pt}
\renewcommand{\bibsection}{}
\makeatletter
\renewcommand{\@biblabel}[1]{[#1]\hfill}
\makeatother
\setlength\parindent{2em}
%\renewcommand{\hth}{\begin{CJK*}{UTF8}{gbsn}}
%\renewcommand{\htss}{\begin{CJK*}{UTF8}{gbsn}}
\renewcommand{\contentsname}{Table of Contents}

\begin{document}

\hyphenpenalty=50000
\makeatletter
\newcommand\mysmall{\@setfontsize\mysmall{7}{9.5}}
\newenvironment{tablehere}
  {\def\@captype{table}}

\let\temp\footnote
\renewcommand \footnote[1]{\temp{\zihao{-5}#1}}

\hypersetup{
  colorlinks=false,
  pdfborder={0 0 0},
}

\thispagestyle{plain}%
\thispagestyle{empty}%
\pagestyle{CjCheadings}

% \begin{table*}[!t]
\vspace {-13mm}


\onecolumn
\zihao{5-}\noindent Group 21 \hfill Lab 1: Gate-Level Verilog \hfill 2024 年 2 月\\
\noindent\rule[0.25\baselineskip]{\textwidth}{1pt}


\begin{center}
    \vspace {11mm}
    {\zihao{2} \heiti \fangsong Lab 1: Gate-Level Verilog }
    
    \vskip 5mm
    
    {\zihao{4}\fangsong Group 21: 陳克盈(112062205)、蔡明妡(112062224)}
\end{center}

\lstset{
    % backgroundcolor=\color{red!50!green!50!blue!50},%程式碼塊背景色為淺灰色
    rulesepcolor= \color{gray}, %程式碼塊邊框顏色
    breaklines=true,  %程式碼過長則換行
    numbers=left, %行號在左側顯示
    numberstyle= \small\ttfamily,%行號字型
    keywordstyle= \color{blue},%關鍵字顏色
    commentstyle=\color{gray}, %註釋顏色
    frame=shadowbox%用方框框住程式碼塊
    basicstyle=\ttfamily\footnotesize,
}
 
\definecolor{improvecolor}{rgb}{0,0.6,0} % 深綠色
\definecolor{declinecolor}{rgb}{0.6,0,0} % 深紅色


%%%%%%%%%%%%%%%%%%%%%%%%%%%%%%%%%%%%%%
\zihao{5}
\vskip 10mm
% \begin{multicols}{1}


%%%%%%%%%%%%%%%%%%%%%%%%%%%%%%%%%%%%%%%%%%
%%%%%%%%%%%%%%%%%%%%%%%%%%%%%%%%%%%%%%%%%%

\tableofcontents
\newpage

\section{NAND Gate to all other gates}
此章節利用 NAND 所組成的所有邏輯閘,都將用來作為後續所有題目的邏輯閘使用。\\
在表示上為求可讀性,將會使用一般的邏輯閘符號來表示。



\section{Full Adder vs. Half Adder}

Half Adder 雖然能夠算出總和以及進位值,但由於缺少了 $cin$  的輸入,\\
導致他只能處理單一位元的加法,這也是為什麼他被稱作是 Half Adder。

\section{Q1: 8-bit ripple carry adder}

根據題目所求,建立八個 Basic Q3 撰寫的 Full Adder,並將他們依據順序將輸入輸出串接在一起,\\
便完成了 8-bit ripple carry adder。

\section{Q2: Decode and execute}

\subsection{Universal gate}
Universal gate 由 $$


\end{document}


